\documentclass[11pt,a4paper]{ivoa}
\usepackage{hyperref}
\input tthdefs
\input gitmeta

\title{VO-DML Tools}

% see ivoatexDoc for what group names to use here; use \ivoagroup[IG] for
% interest groups.
\ivoagroup{Data Models}

\author[https://www.ivoa.net/cgi-bin/twiki/bin/view/IVOA/PaulHarrison]{Paul Harrison}

% \previousversion[????URL????]{????Concise Document Label????}
\previousversion{This is the first public release}


\begin{document}
\begin{abstract}
    The IVOA standard modelling language VO-DML~\cite{2018ivoa.spec.0910L} was developed
    with an associated set of tooling that could be used in the creation of data models
    expressed in VO-DML. The VO-DML standard does not go into any detail about the tooling
    as it describes only the language itself.
    This note attempts to fill that gap by describing the
    development of the tooling, and give an outline of its functionality.
    In addition this document promotes the idea that the VO-DML tools are the de-facto standard way to create the
    schemas for serializations of model instances.
\end{abstract}

\section*{Acknowledgments}
Although the author is the current lead maintainer of the VO-DML Tools, the tools are based on the original contributions
principally of Gerard Lemson and Laurent Bourg\`es, and presumably associated contributions from others working
on SIM-DM \citep{2012ivoa.spec.0503L} and the earlier VO-URP\footnote{\url{https://github.com/glemson/vo-urp}} project.

\section{Introduction}

VO-DML~\cite{2018ivoa.spec.0910L} was developed as a machine-readable language for creating data models.
This was done to promote re-use and rigour in the model definitions, in contrast to the existing practice of
defining models only by description in documentation, which allowed an unconstrained set of possibilities, although
it was common at least to use UML~\cite{std:uml}.
There was a desire to continue to use UML tools to create data models which meant that there was a requirement for
some tooling to convert the UML representation to VO-DML - this requirement was met by the development of the VO-DML tools.
Although the VO-DML tools were not described in the VO-DML standard as it is possible to create VO-DML definitions `by hand' in a text editor,
this is increasingly infeasible to do as the models grow in complexity - at a minimum there needs to be a mechanism to
ensure that any models created do not contain errors and are valid against the VO-DML design criteria.
One of the earliest functionalities of the VO-DML tools was to produce standard documentation and diagrams of the data models,
which again meant that, in practical terms, the use of the VO-DML tools was a requirement when developing a data model, as
the production of such documentation was a requirement of the VO-DML standard.

In addition, one of the main purposes of creating a machine-readable definition of a data model is to be able to
then generate products from this representation.
However, it is at this point that the VO-DML standard does not make any normative requirements - there is some discussion of how this might be done in
in Appendix B of various serialization formats, but not described in sufficient detail to produce interoperable formats.
One of the main aims of this document is to propose that the implementation of these serializations that has been done in the VO-DML
tools is the de-facto standard for how such concrete serializations are created from VO-DML models.

\section{A Brief History of VO-DML Tools}
This section outlines the author's recollection of the timeline of the development of the VO-DML tools. It is
possible, of course, to check the veracity of these recollections in the version control change logs for VO-URP\footnote{\url{https://github.com/glemson/vo-urp/commits/master/}}
and the VO-DML Tools\footnote{\url{https://github.com/ivoa/vo-dml/commits/master/}}

As mentioned in the acknowledgments, the VO-DML tools have their origins in the Simulation Data Model and VO-URP projects, which included
Java code generation from a precursor meta-language that was similar to VO-DML.
In the process of standardizing the VO-DML meta-language, some capabilities of the precursor language were dropped, as well the ability of the tools to create Java
code that would serialize instances of the model described.
The capabilities that the tools possessed at the time
that VO-DML 1.0 was designated a recommendation included;

\begin{itemize}
    \item Being able to transform the output of several UML modelling tools.
    \item Detailed validation of the VO-DML model against the standard.
    \item Creating model documentation as a single HTML page.
\end{itemize}

Starting in early 2021 the author had a requirement to design some new data models
and create services against them.\footnote{This is the Polaris proposal tool \url{https://github.com/orppst} which uses ProposalDM \url{https://ivoa.github.io/ProposalDM/}}
It seemed sensible to try to use VO-DML to define the data models and try to revive the
code generation facilities of VO-URP. The way that the tooling was distributed was also changed
see section \ref{sec:technologies}. The facilities that the current tooling provides are
\begin{itemize}
    \item Creating an XML schema that validates serialized instances of a particular data model
    \item Creating a JSON schema that similarly can be used to validate serialized instances
    \item Creating a TAPSchema representation (an XML serialization of the TAPSchemaDM also defined
          in VO-DML\footnote{\url{https://ivoa.github.io/TAPSchemaDM/}})
    \item Generation of Java code that can create instances of the data model and implements the serializations listed above
    \item Generation of Python code that similarly does the same\footnote{the Python code generation is not as sophisticated as the Java code generation}
    \item Generation of multi page documentation for the data model, which has several advantages over the single page HTML documentation;
    \begin{itemize}
        \item has better navigation.
        \item resizes itself for modern mobile screen sizes.
        \item has a localised diagram for each VO-DML element that shows its connection to the rest of the model.
    \end{itemize}
    \item Being able to author a VO-DML model in VODSL\footnote{\url{https://www.ivoa.net/documents/Notes/VODSL/}} as well as UML.
\end{itemize}

\label{sec:technologies}
\subsection{Technologies used to write the Tools}
As the VO-DML is expressed in XML the VO-DML tools made the early choice to implement all 
of the business logic in XSLT. At the time of publishing VO-VML 1.0 the running of the various
XSLT transformations was orchestrated using the Ant\footnote{\url{https://ant.apache.org}} build tool.
The more recent versions of the VO-DML tools have been rewritten so that the Gradle\footnote{\url{https://gradle.org}} build tool, which
has the following advantages of the previous Ant orchestration;
\begin{itemize}
    \item all of the associated XSLT and other files that are needed for the tools functionality can be packaged as a gradle plug-in
     - this allows use of the tools by a simple textual reference to the plugin, rather than having to copy the required files from the VO-DML repository.
    \item gradle makes it easy to include other dependencies via maven repositories
\end{itemize}

The choice of orchestration tools does look rather `Java-centric' and it reflects the fact that the
initial target of the code generation target was Java. However, this does not mean that the tools are
exclusively of utility to Java programmers - Indeed if only interested in the generated schema and
documentation there is no particular language association (apart from a requirement of an installed
Java Virtual Machine to run the tools). The tools can equally generate python code, and there is
a prototype orchestration that is implemented in Python - however, this orchestration implementation
is not as comprehensive in its functionality as the Gradle orchestration.

\section{Using the VO-DML tools}

This document is deliberately deficient in detail on exactly how to use the VO-DML tools.
This has been done because the intention is that this detail of the use of the tools will be maintained
as on-line documentation at \url{https://ivoa.github.io/vo-dml/} rather than published via the usual IVOA document publishing route.

\section{Serialization}

One of the principle advantages of having a machine readable version of a data model is that it is
possible to further transform the model into other machine readable representations. One of the
principle reasons to create a data model in the first place
is to be able to exchange instances of that data model between
different systems that might be interested in the content of those instances and be sure that
the same interpretation will be made of the instance in each circumstance. Interchange is
done by a process known as serialization where the model instances are transformed into
a form that is easy to transport via some transport mechanism.
This often means that the
model instances are transformed into a textual form for transmission. In recent times
two popular methods of doing this have been XML and JSON - these formats have their own
meta-models that are typically generic enough to be able to express instances of complex
data models.
However, because they are generic they typically cannot express the same semanic
information that is in a data model. In addition these formats, because they were often
defined for interchange
purposes will have their own schema systems to be able to validate instances against
their meta-model.
It is therefore useful to be able to generate schema from a VO-DML model
so that the XML or JSON tools might be able to validate serialization instances of the
data model, as they cannot directly understand the VO-DML.

\subsection{References in serialization}
\subsection{Schema representations}

Depending to the complexity of a particular schema language and serializationformat, there might be several different
styles that can be used to represent data model instance serializations - A possibility that
occurs in XML is whether a particular VO-DML attribute should be represented by an
XML element or an XML attribute - somtimes, for instance, if the VO-DML is a DataType then
only an XML element would be the only formulation capable of being able to represent
the structure in a DataType - therefore the choice between XML element and XML attribute
cannot be a global choice, but can only be made on an attribute by attribute basis.
A purely stylistic choice that can be made with XML for instance is where an XML element
in a sequence that has a multiplicity of greater than one should be `wrapped' by another
element, and although this does not alter any of the semantics of the data model being represented
it often `reads better' to the human eye when looking at such an XML structure.



\section{Standardization}

\section{Future Directions}

It should be noted that the VO-DML tools are in a state of active development and have not reached what the author would consider a 1.0 status.
The development is controlled by using the usual management mechanisms available in GitHub - in particular issues are created for suggested enhancements
and bug fixes and code is controlled via pull requests.

The development of the VO-DML Tools also is a test-bench for future enhancements to the VO-DML language and standard, as new developments will always
be prototyped in the tools before being suggested for the the standardization track.



\appendix
\section{Changes from Previous Versions}

No previous versions yet.
% these would be subsections "Changes from v. WD-..."
% Use itemize environments.


% NOTE: IVOA recommendations must be cited from docrepo rather than ivoabib
% (REC entries there are for legacy documents only)
\bibliography{refs,ivoatex/ivoabib,ivoatex/docrepo}


\end{document}
